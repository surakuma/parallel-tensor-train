% This is samplepaper.tex, a sample chapter demonstrating the
% LLNCS macro package for Springer Computer Science proceedings;
% Version 2.20 of 2017/10/04
%
\documentclass[runningheads]{llncs}
%
\usepackage[utf8]{inputenc}
\usepackage[cmex10,fleqn]{amsmath}
\usepackage{amsfonts,amssymb}
%%\usepackage{amsfonts,amssymb,amsthm}
\usepackage{graphicx}
\usepackage{color}
\usepackage{todonotes}
%%\usepackage{algorithm,algpseudocode}
\usepackage{algorithm}
\usepackage{algorithmic}
%%\usepackage{algcompatible}
\usepackage{subfig}
\usepackage{xspace}
\usepackage{etoolbox}

\definecolor{pdfurlcolor}{rgb}{0,0,0.6}
\definecolor{pdfcitecolor}{rgb}{0,0.6,0}
\definecolor{pdflinkcolor}{rgb}{0.6,0,0}

\usepackage[colorlinks=true,citecolor=pdfcitecolor,urlcolor=pdfurlcolor,linkcolor=pdflinkcolor,pdfborder={0
	0 0}]{hyperref}
%%\usepackage[breaklinks]{hyperref}
%%\usepackage{subcaption}

\usepackage{tikz}
\usetikzlibrary{matrix,snakes, patterns, positioning, shapes, calc, intersections, arrows, fit}


%%\theoremstyle{plain}
%%\newtheorem{lemma}{Lemma}
%%\newtheorem{theorem}{Theorem}
%%\newtheorem{proposition}{Proposition}
%%\newtheorem{corollary}{Corollary}
%%\newtheorem{definition}{Definition}

%%\DeclareMathOperator{\FRT}{FRT}
%%\newcommand{\heteroprioD}{{HeteroPrioDep}\xspace}
\newcommand{\tensor}[1]{\cal\textbf{#1}\xspace}

% Used for displaying a sample figure. If possible, figure files should
% be included in EPS format.
%
% If you use the hyperref package, please uncomment the following line
% to display URLs in blue roman font according to Springer's eBook style:
% \renewcommand\UrlFont{\color{blue}\rmfamily}

\begin{document}
%
\title{Parallel Algorithms for Tensor-Train Decomposition}
%%%%\title{Contribution Title\thanks{Supported by organization x.}}
%%%%%
%%%%%\titlerunning{Abbreviated paper title}
%%%%% If the paper title is too long for the running head, you can set
%%%%% an abbreviated paper title here
%%%%%
\author{}
\institute{}
%%%%\author{First Author\inst{1}\orcidID{0000-1111-2222-3333} \and
%%%%Second Author\inst{2,3}\orcidID{1111-2222-3333-4444} \and
%%%%Third Author\inst{3}\orcidID{2222--3333-4444-5555}}
%%%%%
%%%%\authorrunning{F. Author et al.}
%%%%% First names are abbreviated in the running head.
%%%%% If there are more than two authors, 'et al.' is used.
%%%%%
%%%%\institute{Princeton University, Princeton NJ 08544, USA \and
%%%%Springer Heidelberg, Tiergartenstr. 17, 69121 Heidelberg, Germany
%%%%\email{lncs@springer.com}\\
%%%%\url{http://www.springer.com/gp/computer-science/lncs} \and
%%%%ABC Institute, Rupert-Karls-University Heidelberg, Heidelberg, Germany\\
%%%%\email{\{abc,lncs\}@uni-heidelberg.de}}
%%%%%
\maketitle              % typeset the header of the contribution

\begin{abstract}
The abstract should briefly summarize the contents of the paper in
150--250 words.

\keywords{First keyword  \and Second keyword \and Another keyword.}
\end{abstract}

\section{Introduction}
\label{sec:introduction}
%


%%\begin{table}
%%\caption{Table captions should be placed above the
%%tables.}\label{tab1}
%%\begin{tabular}{|l|l|l|}
%%\hline
%%Heading level &  Example & Font size and style\\
%%\hline
%%Title (centered) &  {\Large\bfseries Lecture Notes} & 14 point, bold\\
%%1st-level heading &  {\large\bfseries 1 Introduction} & 12 point, bold\\
%%2nd-level heading & {\bfseries 2.1 Printing Area} & 10 point, bold\\
%%3rd-level heading & {\bfseries Run-in Heading in Bold.} Text follows & 10 point, bold\\
%%4th-level heading & {\itshape Lowest Level Heading.} Text follows & 10 point, italic\\
%%\hline
%%\end{tabular}
%%\end{table}
%%
%%
%%\noindent Displayed equations are centered and set on a separate
%%line.
%%\begin{equation}
%%x + y = z
%%\end{equation}
%%Please try to avoid rasterized images for line-art diagrams and
%%schemas. Whenever possible, use vector graphics instead (see
%%Fig.~\ref{fig1}).
%%
%%\begin{figure}
%%\includegraphics[width=\textwidth]{fig1.eps}
%%\caption{A figure caption is always placed below the illustration.
%%Please note that short captions are centered, while long ones are
%%justified by the macro package automatically.} \label{fig1}
%%\end{figure}

%%\begin{theorem}
%%This is a sample theorem. The run-in heading is set in bold, while
%%the following text appears in italics. Definitions, lemmas,
%%propositions, and corollaries are styled the same way.
%%\end{theorem}
%%%
%%% the environments 'definition', 'lemma', 'proposition', 'corollary',
%%% 'remark', and 'example' are defined in the LLNCS documentclass as well.
%%%
%%\begin{proof}
%%Proofs, examples, and remarks have the initial word in italics,
%%while the following text appears in normal font.
%%\end{proof}

%
% ---- Bibliography ----
%
% BibTeX users should specify bibliography style 'splncs04'.
% References will then be sorted and formatted in the correct style.
%
 \bibliographystyle{splncs04}
 \bibliography{paralleltt}

\end{document}
